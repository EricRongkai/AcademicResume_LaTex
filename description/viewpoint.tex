\resheading{兴趣爱好及所获荣誉}
  \begin{itemize}[leftmargin=*]
    \item 吉他、摄影、阅读、羽毛球
    \item \textbf{2023}:中国科学技术大学一等学业奖学金
    \item \textbf{2021}:中国科学技术大学一等学业奖学金
    \item \textbf{2021}:中科院合肥物质研究院智能所优秀党员
    \item \textbf{2020}:中国矿业大学(北京)优秀毕业生
    \item \textbf{2018}:中国矿业大学(北京)二等学业奖学金
    \item \textbf{2016}:北京市大学生电子设计大赛省级二等奖
    \item \textbf{2016}:“中国乐势力”全国校园乐队大赛北京赛区冠军
    \item \textbf{2015}:优秀学生干部(院学生会副主席)
    \item \textbf{2015}:优秀学生干部(院学生会文艺部部长)
  \end{itemize}

\resheading{其他在研课题}
\begin{itemize}[leftmargin=*]
  \item \ressubsingleline{步态对称性康复机器人人在环中辅助策略优化}{}{}
  {\small
  \begin{itemize}
    \item 基于已搭建的步态对称性康复机器人平台,以步态时间对称性和空间对称性为优化目标,采用贝叶斯优化等黑盒优化方法对关节力矩辅助策略进行优化以提高偏瘫患者步态表现。
    \item 与已开发的对称步态轨迹生成算法结合,设计机器人关节力-位混合控制策略。
  \end{itemize}
  }
  \item \ressubsingleline{人机共享控制下的辅助机械臂操控}{}{}
  {\small
  \begin{itemize}
    \item 基于所开发的非侵入式体-机交互接口,采用Pybullet机器人物理仿真环境设计辅助机械臂遥操作任务空间。
    \item 基于PPO等深度强化学习算法构建机器智能策略,设计人机混合共享自治模型,提高使用肩部进行人工操控多自由度辅助机器人的效率。
  \end{itemize}
  }
\end{itemize}