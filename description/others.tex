\resheading{实践经历}
  \begin{itemize}[leftmargin=*]
    \item
      \ressubsingleline{中国科学院合肥物质科学研究院-智能所智能感知技术研究中心}{全日制研究生}{2020.09 -- 2024.06}
    \item
      \ressubsingleline{中国科学院自动化研究所-复杂系统管理与控制国家重点实验室}{联合培养}{2017.06 -- 2020.08}
    \item
      \ressubsingleline{北京中科鸿泰医疗机器人有限责任公司}{机器人嵌入式系统实习生}{2019.05 -- 2019.08}
  \end{itemize}
  

% \resheading{资格证书}
%   \begin{itemize}[leftmargin=*]
%     \item
%       \ressubsingleline{全国计算机等级}{二级C,二级C++,三级数据库技术}{2011.03 -- 2012.03}
%     \item
%       \ressubsingleline{全国大学英语}{四级 (549),六级 (464)}{2010.06 -- 2011.12}
%   \end{itemize}
\vspace{-2mm}
\resheading{技能}
\begin{itemize}[leftmargin=*]
  \item \justify{\textbf{编程语言}:Python、C/C++、 C\#、Matlab、LaTex}
\end{itemize}

\vspace{-2mm}
\begin{itemize}[leftmargin=*]
  \item \justify{\textbf{软件开发}:QT、Unity3D、RabbitMQ、ROS、PyBullet、WinForm、Simulink}
\end{itemize}

\vspace{-2mm}
\begin{itemize}[leftmargin=*]
  \item \justify{\textbf{数据分析与可视化}:熟练掌握matplotlib、seaborn、pyqtgraph等静态与动态科研绘图库的使用。
  }
\end{itemize}

\vspace{-2mm}
\begin{itemize}[leftmargin=*]
  \item \textbf{算法研究}:
  {\small
  \begin{itemize}
    \item 熟练掌握常用机器学习算法,及SK-Learn、Pytorch、Tsfresh等算法开发框架;
    \item 熟悉贝叶斯学派下的HMM、KF、EKF、MDP等随机系统时间序列分析与建模等理论知识;
    \item 熟悉DMP、pDMP、GP、GMM、非线性振荡器等机器人模仿学习算法;
    \item 了解LQR、MPC以及强化学习等优化控制算法理论;
    \item 了解AE、VAE、GAIL、GAN等生成式模型理论;
  \end{itemize}}
\end{itemize}

\vspace{-6mm}
\begin{itemize}[leftmargin=*]
  \item \textbf{硬件开发}:
  {\small
  \begin{itemize}
    \item 熟悉机器人系统的整体架构,熟练掌握基于Linux和RTOS平台的嵌入式系统的开发;
    \item 熟悉BLE、2.4G等常用无线通讯的软硬件开发;
    \item 熟练掌握穿戴式机器人电气系统的设计调试以及配套软件的开发,可以独立完成多关节机器人驱动系统的整体设计开发和调试工作。
  \end{itemize}}
\end{itemize}

