\resheading{科研项目}
  \begin{itemize}[leftmargin=*]
    \item \ressubsingleline{基于大数据的自然交互意图理解和智能输入}{国家重点研发计划云计算和大数据专项}{参与}
    {\small
      \begin{itemize}
        \item 针对高位截瘫和截肢患者使用BCI目前难以实现动态环境下连续操控的问题,基于电容式柔性应变传感器网络和IMU,主导设计了一种非侵入式体-机交互界面(SoftBoMI)捕捉人体残余肩部运动以产生连续操控指令控制辅助设备;
        \item 设计了交互接口数据在线解析、校准、数据可视化界面,并基于Unity3D搭建了轮椅操控虚拟仿真环境;
        \item 针对用户肩部动作存在不确定性的问题,以共享自治为主要思想,设计了用户意图推理介入数据解码框架,通过引入先验知识设计指令空间流形,在保证交互界面的动态性能的前提下提高了操控准确性。
      \end{itemize}
    }

    \item \ressubsingleline{脑卒中康复机器人}{国家重点研发计划智能机器人专项}{参与}
    {\small
    \begin{itemize}
      \item 主导设计并带领团队搭建了一套膝关节外骨骼机器人系统,将实时采集的患者健侧步态运动轨迹映射到患侧驱动器执行诱导步态对称性恢复;
      \item 通过非线性频率振荡器与下肢运动进行耦合形式化表示步态周期时间特征,基于机器人模仿学习算法设计了一个编码器-解码器结构的步态轨迹时间序列在线学习与复现算法框架;
      \item 通过采集标准步态数据构建了一个概率形式的机器人步态技能库用于提供机器策略;
      \item 针对用户下肢健侧输入存在不确定性的问题,基于一个共享自治框架实现了用户健侧输入的在线验证消歧和微调以提高机器人系统安全性。
    \end{itemize}
    }

    \item \ressubsingleline{面向机器人交互的柔性应变传感器研制与应用}{安徽省重点研发-长三角}{参与}
    {\small
    \begin{itemize}
      \item 为柔性应变传感器在人-机器人交互方面提供应用场景支撑和验证。
    \end{itemize}
    }

    \item \ressubsingleline{脑损伤康复机器人系统关键技术及康复功能评价方法}{国家自然科学基金}{参与}
    {\small
    \begin{itemize}
      \item 负责人体无线惯性运动捕捉系统的设计与研发工作;
      \item 完成基于MEMS惯性传感器的人体运动捕捉系统的软硬件设计和基于Unity3D的虚拟人实时映射程序开发;
      \item 基于LSTM神经网络提出了一种基于稀疏惯性传感器节点的人体下肢运动重建方法。
    \end{itemize}
    }

    \item \ressubsingleline{可穿戴设备在帕金森慢病管理中的应用}{北京市自然科学基金重点项目}{参与}
    {\small
    \begin{itemize}
      \item 设计了一套可实现多路IMU信号、EMG信号、视频信号的可穿戴同步数据采集系统,用于建立多模态帕金森运动障碍数据集;
      \item 基于BLE和NB-IoT实现了采集数据本地或云端的无线上传;
      \item 采用所开发系统数据采集系统在北京协和医院神经科完成70例典型帕金森患者的运动信息采集,构建了帕金森症状量化评估数据集(包含上肢运动能力,步态运动信息,视频信息以及声音等);
      \item 对实验采集所得数据进行标准化处理和数据挖掘,结合领域专家知识进行特征工程,基于RBF-SVM实现了帕金森患者的步态障碍和运动能力的量化评估与异常步态检测。
    \end{itemize}
    }
    \item \ressubsingleline{基于sEMG和FES的上肢康复机器人自适应主动控制方法研究}{国家自然科学基金青年基金}{参与}
    {\small
    \begin{itemize}
      \item 负责6-DOF外骨骼上肢康复机器人的通讯与运动控制嵌入式系统的研发和调试;
      \item 基于Unity3D为6-DOF和2-DOF上肢康复机器人开发了虚拟现实康复训练系统。
    \end{itemize}
    }
    \item \ressubsingleline{智能轮椅助行车技术研发与产业化}{中国科学技术大学“双创基金”}{主持}
    {\small
    \begin{itemize}
      \item 负责项目总体构思和申报工作;
      \item 围绕实验室基础和现有研究成果布局未来可能的产业化方向,开展初步市场需求调研。
    \end{itemize}
    }
  \end{itemize}
