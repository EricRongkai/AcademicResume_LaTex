\resheading{论文及专利}
学术论文发表:
\vspace{-2mm}
  \begin{itemize}[leftmargin=*]
    \item 
    \justify{\textbf{R. Liu}, Q. Song, T. Ma, and H. Pan, ``Toward Remapping Residual Movement of Shoulder: A Soft Body-Machine Interface,'' \textit{IEEE Transaction on Neural System and Rehabilitation (T-NSRE)}., Under Review. (IF:4.9)}
  \end{itemize}
  \vspace{-2.6mm}

  \begin{itemize}[leftmargin=*]
    \item 
    \justify{\textbf{R. Liu}, T. Ma, and N. Yao et al., “Adaptive Symmetry Reference Trajectory Generation in Shared Autonomy for Active Knee Orthosis,” \textit{IEEE Robotics and Automation Letters (RA-L) with IROS 2023}, vol. 8, no. 6, pp. 3118–3125, Jun. 2023. (IF:5.2)}
  \end{itemize}

  \vspace{-2.6mm}
  \begin{itemize}[leftmargin=*]
    \item 
    \justify{X. Zhao, \textbf{R. Liu}, and T. Ma et al, “Real-time Gait Phase Estimation Based on Multi-source Flexible Sensors Fusion,”  in 2023 3rd International Conference on Robotics and Control Engineering(RobCE2023), Nanjing, China: ACM, May. 2023, in production.}
  \end{itemize}

  \vspace{-2.6mm}
  \begin{itemize}[leftmargin=*]
    \item 
    \justify{L. Tong, \textbf{R. Liu}, and L. Peng, “LSTM-Based Lower Limbs Motion Reconstruction Using Low-Dimensional Input of Inertial Motion Capture System,” \textit{IEEE Sensors Journal}, vol. 20, no. 7, pp. 3667–3677, Apr. 2020.(IF:4.3, 导师一作)}
  \end{itemize}

  \vspace{-2.6mm}
  \begin{itemize}[leftmargin=*]
    \item 
    \justify{\textbf{R. Liu}, L. Peng, and L. Tong et al, “A Novel Method for Parkinson’s Disease Classification and Dyskinesia Quantification Using Wearable Inertial Sensors,” in 2019 IEEE 9th Annual International Conference on CYBER Technology in Automation, Control, and Intelligent Systems (CYBER), Suzhou, China: IEEE, Jul. 2019, pp. 1022–1026.}
  \end{itemize}

  \vspace{-2.6mm}
  \begin{itemize}[leftmargin=*]
    \item 
    \justify{\textbf{R. Liu}, L. Peng, and L. Tong et al., “The Design of Wearable Wireless Inertial Measurement Unit for Body motion Capture System,” in 2018 IEEE International Conference on Intelligence and Safety for Robotics (ISR), Shenyang: IEEE, Aug. 2018, pp. 557–562.}
  \end{itemize}

  \vspace{-2.6mm}
  \begin{itemize}[leftmargin=*]
    \item 
    \justify{Y. Wang, Q. Song, T. Ma, Y. Chen, H. Li, and \textbf{R. Liu}, “Transformation classification of human squat/sit-to-stand based on multichannel information fusion,” \textit{International Journal of Advanced Robotic Systems}, vol. 19, no. 4, Jul. 2022.(IF:2.3)}
  \end{itemize}

  \vspace{2mm}
  已授权发明专利: 
  \vspace{-1mm}
  \begin{itemize}[leftmargin=*]
    \item 
    \justify{彭亮,侯增广,{\CJKfontspec{微软雅黑} 刘镕恺}《一种用于帕金森病症状量化评估的可穿戴设备》}
  \end{itemize}
  \vspace{-3mm}
  \begin{itemize}[leftmargin=*]
    \item 
    \justify{彭亮,侯增广,{\CJKfontspec{微软雅黑} 刘镕恺}《基于支持向量机的帕金森病人运动障碍量化及识别方法》}
  \end{itemize}
  \vspace{2mm}
  在投发明专利: 
  \vspace{-1mm}
  \begin{itemize}[leftmargin=*]
    \item 
    \justify{宋全军,{\CJKfontspec{微软雅黑} 刘镕恺},马婷婷《一种基于共享自治系统的步态对称性康复机器人轨迹规划算法》}
  \end{itemize}
  \vspace{-3mm}
  \begin{itemize}[leftmargin=*]
    \item 
    \justify{宋全军,{\CJKfontspec{微软雅黑} 刘镕恺},马婷婷《一种用于高位截瘫患者的非侵入式柔性人机交互接口》}
  \end{itemize}
  \vspace{2mm}