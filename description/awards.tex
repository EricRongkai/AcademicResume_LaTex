\resheading{科研项目}
  \begin{itemize}[leftmargin=*]
    \item \ressubsingleline{基于大数据的自然交互意图理解和智能输入}{国家重点研发计划云计算和大数据专项}{参与}
    {\small
      \begin{itemize}
        \item 围绕高位截瘫和上肢残疾患者操控辅助机器人的需求,基于电容式柔性应变传感器网络和IMU捕捉患者肩部残余运动,主导并设计了一种非侵入式体-机交互界面用于产生连续指令操纵电动轮椅和控制光标。
        \item 针对用户肩部动作存在不确定性的问题,设计了一种基于共享控制系统的用户意图推理介入交互数据解码算法,在保证交互界面的动态性能的前提下提高了操控准确性。
      \end{itemize}
    }

    \item \ressubsingleline{脑卒中康复机器人}{国家重点研发计划智能机器人专项}{参与}
    {\small
    \begin{itemize}
      \item 围绕脑卒中患者步态对称性康复需求,主导设计并带领团队搭建了一套完整的膝关节外骨骼机器人系统,将实时采集的患者健侧步态运动轨迹映射到患侧驱动器执行进而实现步态对称性恢复。
      \item 针对用户下肢健侧输入动作存在不确定性的问题,基于机器人模仿学习算法将步态时间序列编码到一个固定长度的向量空间。
      \item 通过采集健康人群的步态运动轨迹,构建了机器人步态技能库,并基于一个共享自治系统实现了用户健侧输入的在线验证和微调以保证机器人系统的整体安全性和轨迹个性化。
    \end{itemize}
    }

    \item \ressubsingleline{面向机器人交互的柔性应变传感器研制与应用}{安徽省重点研发-长三角}{参与}
    {\small
    \begin{itemize}
      \item 为柔性应变传感器在人-机器人交互方面提供应用场景支撑
    \end{itemize}
    }
    \item \ressubsingleline{脑损伤康复机器人系统关键技术及康复功能评价方法}{国家自然科学基金}{参与}
    {\small
    \begin{itemize}
      \item 负责人体无线惯性运动捕捉系统的设计与研发工作,完成基于MEMS惯性传感器的可穿戴运动捕捉节点软硬件系统开发设计以及基于Unity3D的虚拟人姿态解算与虚拟现实三维重构程序开发。
      \item 基于LSTM神经网络提出了一种基于稀疏节点的人体下肢运动重建方法。
    \end{itemize}
    }
    \item \ressubsingleline{基于 sEMG 和 FES 的上肢康复机器人自适应主动控制方法研究}{国家自然科学基金青年基金}{参与}
    {\small
    \begin{itemize}
      \item 负责五自由度外骨骼上肢康复机器人的嵌入式运动控制系统的研发
      \item 完成基于Unity3D的五自由度以及二自由度上肢康复机器人虚拟现实仿真和训练系统的开发。
    \end{itemize}
    }
    \item \ressubsingleline{可穿戴设备在帕金森慢病管理中的应用}{北京市自然科学基金重点项目}{参与}
    {\small
    \begin{itemize}
      \item 围绕帕金森病患者状态监测需求,设计了一套可实现多路IMU运动信号、EMG信号、视频信号的可穿戴同步数据采集系统用于开展前期研究,并基于NB-IoT以及BLE实现了采集数据无线上传。
      \item 采用所开发系统在北京协和医院神经科完成60例典型帕金森患者的运动信息采集.
      \item 对实验采集所得数据进行标准化处理并进行数据挖掘,结合专家知识设计特征工程,基于RBF-SVM实现了帕金森患者的步态障碍和运动能力的量化评估与异常步态检测。
    \end{itemize}
    }
    \item \ressubsingleline{智能轮椅助行车技术研发与产业化}{中国科学技术大学“双创基金”}{主持}
    {\small
    \begin{itemize}
      \item 负责项目总体构思和申报工作,围绕实验室基础和现有研究成果布局未来可能的产业化方向,开展初步市场需求调研。
    \end{itemize}
    }
  \end{itemize}